\documentclass{apostila}

\title{GO}
\author{Neni}
\date{\today}

\begin{document}
\chapter{O que é GO}
\begin{flushleft}

    Golang é uma linguagem de programação criada pela Google para resolver seus problemas\cite[1]{cod3r}. Desenvolvida por Robert Griesemer, Rob Pike e Ken Thompson em 11/2009\cite[4]{cod3r}.
    \\
    Caracerísticas:
    \begin{itemize}
        \item Linguagem compilada\cite[1]{cod3r};
        \item Fortemente Tipada\cite[4]{cod3r};
        \item Código livre e aberto\cite[4]{cod3r};
        \item Tempo de compilação otimizado\cite[4]{cod3r};
        \item Garbage Collector\cite[4]{cod3r};
        \item Linguam mínima\cite[4]{cod3r};
        \item C-like\cite[4]{cod3r};
        \item Não é OO, porém possui métodos e interfaces\cite[4]{cod3r};
        \item Concorrência com goroutines\cite[4]{cod3r};
    \end{itemize}
\end{flushleft}









%%%%%%%%%%%%%%%%%%%%%%%%%%%%%%%%%%%%%%%%%%%%%%%%%%%%%%%%

\chapter{Configurando ambiente}
\section{Instalando GO}
Seguir as instruções de acordo com seu sistema operaional em: \link{https://golang.org/dl/}{Download}.
\dica{
    GOPATH é a variável de ambiente que indica o local dos códigos go\cite[15]{igo}.\\
    GOROOT é a variavel de ambiente que indica o local de instalação do go (compilador e ferramentas)\cite[7]{cod3r}
}

%\codefile{go}{exemplos/teste.go}

\section{Editores e IDEs}
\begin{itemize}
    \item \link{https://code.visualstudio.com/}{Visual Studio Code}: Linux, Windows e Mac;
    \item \link{https://www.vim.org/download.php}{Vim}: Linux, Windows e Mac\footnotemark[1];
    \item \link{https://github.com/neovim/neovim/wiki/Installing-Neovim}{NeoVim}: Linux, Windows e Mac\footnotemark[1];
    \item \link{https://atom.io/}{Atom}: Linux, Windows e Mac.
\end{itemize}
\footnotetext[1]{\link{https://github.com/fatih/vim-go}{Plugin}}








%%%%%%%%%%%%%%%%%%%%%%%%%%%%%%%%%%%%%%%%%%%%%%%%%%%%%%%%

\chapter{Ferramentas Go}

\begin{table}[h]
\centering
\caption{Ferramentas go}
\vspace{0.5cm}
\begin{tabular}{r|l}
 
Comando 	& Descrição	\\
\hline
build 		& Compila pacote main       \\
clean		&  \\
doc 		& go doc [package] exibe documentação do pacote\cite[78]{goia}.        \\
    env 		& Informações sobre ambiente do go instalado\cite[9]{cod3r}.       \\
bug 		&        \\
fix 		&        \\
fmt 		& Formata o arquivo .go\cite[77]{goia}       \\
generate 	&        \\
get 		& Download de pacotes a partir do caminho de importação\cite[9]{cod3r}. Usado para baixar do github\cite[9]{cod3r}.       \\
install 	&        \\
list 		&        \\
run 		& Compila e executa pacote main\cite[76]{goia}       \\
test 		&       \\
tool 		&       \\
version 	&       \\
vet 		& Verifica erros no código\cite[9]{cod3r}.  \\
c			&       \\
buildmode 	&       \\
cache 		&       \\
filetype 	&       \\
gopath 		&       \\
enviroment 	&       \\
importpath 	&       \\
packages    &       \\
testflag    &       \\
testfunc    & 
 
\end{tabular}
\end{table}

\dica{Digitando \texttt{go} é possível ver todas as ferramentas da linguagem.}
\dica{O comando \texttt{godoc -http=:6060} cria servidor web com documentação de pacotes\cite[78]{goia}}









%%%%%%%%%%%%%%%%%%%%%%%%%%%%%%%%%%%%%%%%%%%%%%%%%%%%%%%%

\chapter{Funções e packages}
\section{Estrutura de um arquivo GO}
A estrutura comum de um arquivo possui 3 partes: Declaração de pacote, importe de pacotes/bibliotecas e funções - nessa ordem. Abaixo exemplos comentados (\texttt{//} ou  \texttt{/* */} comentam códigos Go).

\begin{lstlisting}[language=go,caption=Estrutura de um arquivo.go]
// declaração de pacote
package main

// importe de pacotes/bibliotecas
import (
    "fmt"
    "os"
)

// Funções
func main(){
    // algoritmo
}

func abcde(){
    // algoritmo
}
\end{lstlisting}



\section{Funções}
Função é um bloco de código nomeado com algoritmo que pode tanto receber (parâmetros) quanto retornar 0 ou mais valores\cite[15]{cod3r}. Por pasta/pacote não devem haver funções com nome repetido\cite[8]{cod3r} - Go não possui sobrecarga de métodos.

\begin{lstlisting}[language=go,caption=Funções]
package main

import "fmt"

func main(){
    // fmt.Print exibe resultado da função divideValor
    fmt.Print(divideValor(4, 2))

    // fmt.Print exibe resultado da função somaValores
    fmt.Print(somaValores(4, 2))
}

// Recebe dois valores e retorna um
func divideValor(n1 int, n2 int) float64 {
    // Retorna resultado da divisão
    return n1/n2
}

// Recebe quantidade indefinida de valores (função variática) e retorna um
func somaValores(nums ...int) float64 {
    // Variável incrementada com os valores
    total := 0

    // Loop pela quantidade de numeros passados
    for _, num := range nums {
        // total = total + num
        total += num
    }

    // Retorna total
    return total
}
\end{lstlisting}

\section{Packages}
Pacotes são a maneira de Go organizar e reutilizar código\cite[18]{igo}. Há dois tipos de programa em Go: executáveis e bibliotecas\cite[18]{igo}.

\begin{itemize}
    \item Executaveis necessitam de um arquivo com pacote main declarado e uma função main\cite[8]{cod3r}.
    \item Bibliotecas são arquivos que podem ser reutilizados emqualquer projeto quando importados. Funções, variáveis e tipos  podem ser visíveis caso seu nome comece com letra maiúscula\cite[106]{igo}.
\end{itemize}

\begin{lstlisting}[language=go,caption=Executável]
package main

func main(){
    // ...
}
\end{lstlisting}

\begin{lstlisting}[language=go,caption=Biblioteca em \$GOPATH/src/github.com/nenitf/abcdef]
// Nome do package
package abcdef

// Função exposta (letra inicial maiúscula)
// Pode ser usada por um exeutável
func ProximaLetraAlfabeto(){
    // ...
}

// Função oculta (letra inicial minúscula)
// Só pode ser usada dentro da própria bilioteca
func loopAlfabeto(){
    // ...
}
\end{lstlisting}

\begin{lstlisting}[language=go,caption=Executável utilizando biblioteca abcdef]
package main

// importando pacote
import "github.com/nenitf/abcdef"

func main(){
    absdef.ProximaLetraAlfabeto()
}
\end{lstlisting}
Go possui uma série de pacotes próprios que podem ser importados, no capítulo \ref{main lib} os mais usuais são explicados.










%%%%%%%%%%%%%%%%%%%%%%%%%%%%%%%%%%%%%%%%%%%%%%%%%%%%%%%%

\chapter{Documentando}
Para documentar pacotes, funções, variáveis e tipos visíveis, basta comentar imediatamente acima da declaração\cite[80]{goia}.

\begin{lstlisting}[language=go,caption={Documentando pacote, variaveis, funcoes e tipos}]
package ccord

// GMStoUTM converte coordenada GMS para UTM
func GMStoUTM() {
	// ...
}
\end{lstlisting}

\dica{Pode criar arquivo doc.go no diretório do seu pacote únicamente para documentá-lo\cite[81]{goia}}.

\begin{lstlisting}[language=go,caption=Documentando pacotes]
/*
    Pacote responsável por
    calculos com coordenadas
*/
package ccord
\end{lstlisting}












%%%%%%%%%%%%%%%%%%%%%%%%%%%%%%%%%%%%%%%%%%%%%%%%%%%%%%%%

\chapter{Main packages}
\label{main lib}
Funções mais usuais dos pacotes mais comuns da biblioteca padrão do Go.
\section{fmt}
\section{os}
\section{flag}
\section{http}
\section{log}
\section{strings}

\chapter{Variáveis e tipos}
\chapter{Arrays e slices}
\chapter{Fluxo de controle}
\chapter{Loops}
\chapter{Estruturas e interfaces}
\chapter{Ponteiros}
\chapter{Concorrência}
\chapter{Testes}
\chapter{Benchmarks}


\end{document}
