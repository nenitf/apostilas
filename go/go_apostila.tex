\documentclass{apostila}
% cp ~/dev/dotfiles/cpfiles/latex/apostila.cls apostila.cls

\title{GO}
\author{Neni}
\date{\today}

\begin{document}
\chapter{O que é GO}
\begin{flushleft}

    Golang é uma linguagem de programação criada pela Google para resolver seus problemas\cite[1]{cod3r}. Desenvolvida por Robert Griesemer, Rob Pike e Ken Thompson em 11/2009\cite[4]{cod3r}.
    \\
    Caracerísticas:
    \begin{itemize}
        \item Linguagem compilada\cite[1]{cod3r};
        \item Fortemente Tipada\cite[4]{cod3r};
        \item Código livre e aberto\cite[4]{cod3r};
        \item Tempo de compilação otimizado\cite[4]{cod3r};
        \item Garbage Collector\cite[4]{cod3r};
        \item Linguam mínima\cite[4]{cod3r};
        \item C-like\cite[4]{cod3r};
        \item Não é OO, porém possui métodos e interfaces\cite[4]{cod3r};
        \item Concorrência com goroutines\cite[4]{cod3r};
    \end{itemize}
\end{flushleft}

\chapter{Configurando ambiente}
\section{Instalando GO}
Seguir as instruções de acordo com seu sistema operaional em: \link{https://golang.org/dl/}{Download}.
\dica{
    GOPATH é a variável de ambiente onde indica aonde estão os códigos go\cite[15]{igo}.
}
\dica{
    O comando \texttt{go env} lista todas variáveis de ambiente.
}


\section{Editores e IDEs}
\begin{itemize}
    \item \link{https://code.visualstudio.com/}{Visual Studio Code}: Linux, Windows e Mac;
    \item \link{https://www.vim.org/download.php}{Vim}: Linux, Windows e Mac\footnotemark[1];
    \item \link{https://github.com/neovim/neovim/wiki/Installing-Neovim}{NeoVim}: Linux, Windows e Mac\footnotemark[1];
    \item \link{https://atom.io/}{Atom}: Linux, Windows e Mac.
\end{itemize}
\footnotetext[1]{\link{https://github.com/fatih/vim-go}{Plugin}}
\end{document}
